\chapter{Návrh}
\label{navrh}

Cílem této kapitoly je návrh simulátoru, jeho rozhraní pro komunikaci s periferiemi a mikrokontroléry a knihovna implementující mikrokontrolér MSP430.
Je zde také zmíněno grafické uživatelské rozhraní simulátoru.

\section{Návrh simulátoru}

Simulátor je jádrem celého projektu. Umožní propojení jednotlivých simulovaných komponent (jednoho mikrokontroléru a více periferií), řízení simulace
a přeposílání zpráv mezi komponentami.

Simulace bude založena na formalismus spojoveného DEVS (coupled DEVS). Bude se tedy jednat o diskrétní simulaci (discrete-event simulation) 
modelující chod systému jako diskrétní sled událostí v čase. Každá událost se objevuje v konkrétním časovém okamžiku a značí změnu stavu v systému.
Mezi dvěma po sobě jdoucími událostmi se nepředpokládá žádná změna systému, takže simulace může přímo přeskočit v čase z jedné události na druhou.

\subsection{Komponenty}

Komponenty jsou navrženy jako samostatné moduly s pevně daným rozhraním, pomocí kterého komunikují s ostatními komponentami simulace. Návrh je
tak velmi obecný a umožňuje přidávání dalších rozšiřujících komponent.
Návrh počítá se dvěma typy komponent:

\begin{itemize}
\item \textbf{Periférie} - Je jakákoliv simulační komponenta, která umí reagovat na zprávy přijímané od jiných periférií, generovat nové zprávy a měnit svůj vnitřní stav
na základě simulačního času.
\item \textbf{Mikrokontrolér (MCU)} - Jedná se o speciální případ periférie, která obsahuje paměť, registry a kód programu, který vykonává.
To umožní řídit simulaci na základě instrukcí, obsahu paměti a registrů daného mikrokontroléru.  V rámci celé simulace je jen jedna instance mikrokontroléru. 
\end{itemize}

Z pohledu DEVS formalismu je komponenta definovaná jako: TODO

Každá komponenta (periférie) tak bude schopna následujícího chování:

\begin{itemize}
\item Změnit svůj vnitřní stav na základě změny simulačního času.
\item Změnit svůj vnitřní stav na základě externí události (typicky na základě zprávy od jiné komponenty).
\item Změnit svůj vnitřní stav na základě externí události přijaté na vstupní port (typicky na základě zprávy od jiné komponenty).
\item Generovat zprávy na výstupní port.
\end{itemize}

Mikrokontroléry navíc oproti periferiím poskytují následující informace:

\begin{itemize}
\item Obsah paměti a registrů.
\item Disassemblovaný kód běžícího programu.
\end{itemize}

Vstupní a výstupní porty definované ve formalismu DEVS budou představovat jednotlivé piny komponent. Spojení mezi porty pak značí spojení jednolivých
pinů a zprávy posílané mezi porty budou reprezentovat aktuální napětí na pinech.

Na obrázku .... lze vidět ukázku propojení dvou komponent (mikrokontroléru MSP430 a periférie - LED diody). Pokud program mikrokontroléru vygeneruje na svůj 
výstupní port hodnotu 3.3 (tzv. 3.3 voltu), bude tato zprávy předána na vstupní LED diody, která na jejím základě změní svůj vnitřní stav (tzv. rozsvítí se).

\section{Návrh komponenty simulující MCU MSP430}

Komponenta simulující mikrokontrolér MSP430 je klíčovou komponentou projektu. Bude umožňovat nahrání uživatelského programu a jeho následnout simulaci na základě
vykonávání instrukcí a běhu svých dalších interních komponent. Umožní grafickému uživatelskému rozhraní přístup ke své paměti a registrům, čímž umožní krokování
programů. Bude také poskytovat další ladící informace jako je například umístění lokálních a globálních proměnných v paměti nebo v registrech.

Na obrázku ... lze vidět základní schéma MSP430 komponenty. V této kapitole jsou jednotlivé části tohoto schématu podrobněji rozebrány s důrazem na jejich funkce a 
postavení v rámci celé MSP430 komponenty.

